\documentclass[12pt]{article}

\usepackage{sbc-template}

\usepackage{graphicx,url}

%\usepackage[brazil]{babel}   
\usepackage[latin1]{inputenc}  

     
\sloppy

\title{Ideia Mais que Genial \\ A Turma Do Fund�o}

\author{Higor Oliveira\inst{1}, Rafael Viana\inst{1}, Ramon Santos\inst{1} }


\address{Sistemas de Informa��o -- Universidade Federal do Mato Grosso do Sul
  (UFMS)\\
  Caixa Postal --- 79400-000 -- Mato Grosso do Sul -- MS -- Brazil
  \email{\{higor.oliveira,rafael.viana,ramon.santos\}@aluno..ufms.br}
}

\begin{document} 

\maketitle

\begin{abstract}
  This meta-paper describes the style to be used in articles and 
\end{abstract}
     
\begin{resumo} 
  Este meta-artigo descreve 
\end{resumo}


\section{Introdu��o}

\section{Objetivo}

\begin{itemize}
	\item listar os objetivos
\end{itemize}


\section{Metodologia}

Listar como ser�o os procedimentos para gerar o resultado final.

A partir dos dados quais ser�o os passos para chegar ao objetivo.

Qual a estrutura de dados seria utilizada

Quais t�cncias de pr�-processamento de dados seriam implantados

Quais os par�metros seriam utilizados no projeto

\textbf{Quais t�cncias seriam utilizadas}


\section{Conclus�o}



\bibliographystyle{sbc}
\bibliography{trabalho}

\end{document}
